\chapter{\uppercase {Conclusion and Future recommendation}}

Although surface wave propagation has been actively studied for the past four decades in the form of surface plasmon polaritons and plasma waves in the optical and terahertz frequency regimes respectively, the engineered devices that make use of the interesting properties are still in their infancy. Lately, there has been a huge interest in  

ngineered
Planar stratified media serve as the basic structure for a plethora of RF, microwave and
recently THz components. One of the most suitable and efficient numerical methods for
their EM analysis is the Integral Equation method combined with the Method of Moments
discretization technique. Among the different Integral Equation formulations, Mixed Potential
Integral Equations (MPIE) serve as one of the most competitive due to the lower order
singularities that appear in the kernels of the integrals.
The target of this thesis was to provide a general framework for developing the appropriate
Green Functions (GFs) for each multilayered structure. Although this is a classical CEM
problem where many researchers have already faced most of the difficulties, new designs that
include very thin conductive sheets have to be accommodated. Recent technological advances
in the area of thin conductive sheets, like the fabrication of the one atom thick graphene,
have provided a platform for new EM devices, especially in the area of THz. Properties like
plasmon propagation at the THz range or field amplification enhance the scientific interest.
The extremely small thickness, up to molecular scale for the case of graphene, deteriorates
the performance of the current methods. For this reason, a general method based on the
propagator matrix technique, enhanced so as to include the conductive sheets, is provided
in Chapter 4. Even arbitrary tensorial conductivity, a very challenging situation that comes
under the simultaneous electric and magnetic bias of graphene, can be taken into account
with this approach. GFs in the spectral domain for the EM fields and the scalar and vector
potentials have been obtained for cases with high scientific and technological interest. The
validity of the results, especially the position of the TM and TE poles, that can be derived with
alternative techniques, has been verified for the cases a literature reference exists. Finally, the
spectral domain mixed potential GFs serve as the first necessary step in order to employ the
efficient MPIE method in such problems.
The spatial domain counterparts of these GFs have to be calculated efficiently and in
an error-controllable manner through the Sommerfeld integrals. Novel techniques based on
specialized Double Exponential (DE) quadrature rules and the Weighted Averages (WA)
algorithm have been presented in Chapter 6. The DE rule has been revisited in order to
account for adaptivity and efficient error estimation as discussed in Chapters 5 and 6. The
algorithm outperforms in terms of speed the reference method, that is the detour of the
integration path of the Sommerfeld integral. It manages to provide accurate results up to the
predefined accuracy level. If average accuracy is enough for certain applications, a very fast
result can be provided, while at the cost of relatively increased computational resources, even
numerically exact results can be available. It should always be mentioned that the proposed
method does not require a detour of the integration path into the complex plane to recover
