\chapter{\uppercase {Conclusion and Future recommendations}}

In this work, structures that support surface wave propagation were studied in the terahertz as well as the  optical frequency domain. The designs that were discussed provide an ideal outlet for realization of miniaturized communication devices. A concrete theory was presented regards to the understanding of the working principles of optical nanoantennas. The essential requirement for surface plasmons to exist is the negative real part of the dielectric function of metals in the optical frequency range. In the terahertz frequency region, this necessary condition is satisfied by a thin plasma region formed at the interface of a semiconductor heterostructure.

The difficulties of electromagnetic analysis of extremely thin layers using commercial software were discussed. An integral equation based technique was presented that computes the fields via transmission line Green functions of the structure by which even infinitesimally thin sheets of a lossy material, defined by a complex valued surface conductivity, can be conveniently solved for using simple circuit and transmission line techniques. The mixed potentials formulation was adopted due to weaker singularities that show up in the integral kernels.

The dispersion curves were plotted both for surface plasmons and 2D plasma waves. In the former case, analytical solutions of the dispersion relation were available while in the latter case, sophisticated complex-valued root-finding methods were invoked. The results illustrated the subwavelength properties of both waves in their respective domains of frequencies. In this regard, a nanoscopy imaging scheme was proposed that utilized the 2D plasma waves to illuminate a sample and yield super-resolution.

\section*{Recommendations for future work}
%
%
The research in the terahertz frequency region is currently witnessing an exponential level of interest, primarily due to lack of efficient sources and sensors. Plasma based semiconductor structures have been declared to be the leading candidates to fill this void. However, with current availability of materials and semiconductor technology, reasonable levels of efficiencies are only attainable at low temperatures. Extensive efforts are currently been made to explore materials such as perovskites and dichalcogenides that show promise for device operability at higher temperatures.

Throughout this dissertation, there are various aspects that can be greatly improved. The surface conductivity was assumed as a scalar quantity, although in a true sense, especially in the presence of external electric or magnetic fields, it should be represented by a tensor. The discovery of quantum Hall effect that led to two Nobel prizes in physics, was made possible through the observation of a 2DEG upon which an external magnetic field was applied.

In Chapter III, the Green functions were computed only for a 2DEG plasma sheet suspended in freespace. The procedure can easily be extended to incorporate the transistor structure consisting of a semiconductor heterostructure and a gate terminal.

The root-finding technique discussed in Chapter IV currently involves a lot of guess work in determining whether the poles are proper or improper. It can be improved by systematically incorporating the knowledge of Riemann sheets.
