\documentclass[11pt]{article}
% Horizontal Magnetic Dipole over a lossy half-space
\usepackage[utf8]{inputenc} % Use it to include other characters than ABC
\usepackage[T1]{fontenc}
\usepackage[cmex10]{amsmath}
\usepackage{calc}
% \usepackage{systeme} % For system of equations
\usepackage{amsfonts} % to load math symbols
\usepackage{mdwmath}
\usepackage{commath}
\usepackage{mdwtab}
\usepackage{hyperref}
\usepackage{physics} % For using the oridnary derivative nomenclature
\usepackage{datetime} % Insert date and time
\usepackage[letterpaper]{geometry}
\geometry{verbose,tmargin=1.25in,bmargin=1.25in,lmargin=1.4in,rmargin=1.15in}
\usepackage[nodisplayskipstretch,doublespacing]{setspace}
\setstretch{1.5}
\usepackage{etoolbox}
%% Nicely set the spacing between equations and text
\AtBeginDocument{%
\setlength\abovedisplayskip{4pt}
\setlength\belowdisplayskip{4pt}
\setlength\abovedisplayshortskip{4pt}
\setlength\belowdisplayshortskip{4pt}
}
% \abovedisplayskip=12pt
% \belowdisplayskip=12pt
% \abovedisplayshortskip=0pt
% \belowdisplayshortskip=7pt
% \appto{\normalsize}{\zerodisplayskips}
% \appto{\small}{\zerodisplayskips}0pt
% \appto{\footnotesize}{\zerodisplayskips}
\usepackage{tocloft}
% \usepackage[rm, tiny, center, compact]{titlesec}
\usepackage{indentfirst}
\usepackage{tocvsec2}
% \usepackage[titletoc]{appendix}
% \usepackage{appendix}
% \usepackage{tamuconfig}
%
% \usepackage{rotating}
\usepackage{graphicx}
\usepackage{pgfplots}
\usepackage{tikz}
\usepackage{standalone}
\usepackage[americanresistors,americaninductors]{circuitikz}
\usepackage{tikz-dimline} % For dimensional drawing
\usetikzlibrary{positioning}
\usetikzlibrary{arrows}
\usepackage{subfig}
% The following is done to hide ugly color boxes around the links
\usepackage{xcolor}
\hypersetup{
colorlinks,
linkcolor={red!50!black},
citecolor={blue!50!black},
urlcolor={blue!80!black}
}
% pdflatex -synctex=-1
% \usepackage{mathptmx} % Times new Roman
\usepackage{times}
%
% ------------------------------- Useful Tricks Learnt
% Use ={}& to align subequations to the left

% Use = for single equations

% Use &= for split equations

% Use commath package to properly write differential operators and derivatives.

% Use \int\limits to nicely put integral limits

% For long equations, use align environment with \notag\\ as a linebreak.

% To hide section numbers, place an asterisk after the section, e.g., \section*{}

% Put comments % in between the lines in order to avoid forming a new paragraph.

% To enter special characters into Inkspace figures, use Ctrl+U and then enter       the unicode. e.g., for \times symbol, the unicode is U+0D7. So the key entry would be Ctrl+U U+0d7 and then press enter.

% Put \eqref instead or \ref to reference equations. This will automatically put parantheses around the eq. number. amsmath package required.
%
% ----------------- To compile with references use the following order in Shell"
% 1. pdflatex filename.tex
% 2. bibtex filename (no extension)
% 3. bibtex filename (no extension)
% 4. pdflatex filename.tex
% -----------------

% Personal definitions
% Operators
\renewcommand{\v}[1]{\mathbf{#1}} % vectors
\newcommand{\ti}[1]{\tilde{#1}} % spectral representation

% Symbols
\renewcommand{\O}{\omega}  % omega
\newcommand{\E}{\varepsilon}  % epsilon
\renewcommand{\u}{\mu}  % mu
\newcommand{\p}{\rho}  % rho
\newcommand{\x}{\times}  % times
\renewcommand{\inf}{\infty}  % infinity
\newcommand{\infint}{\int\limits_{-\inf}^\inf} % integral by R
\renewcommand{\del}{\nabla}  % nabla operator
\renewcommand{\^}{\hat}  % unit vector
\newcommand*\diff{\mathop{}\!\mathrm{d}} % Define differential operator



\begin{document}
\title{\textsc{Current on a planar Dieletric plate}\\}
\date{\footnote{Last Modified: \currenttime, \today.}}
\maketitle
%
We apply the surface equiavalence theorem to find the electric and magnetic currents on the surface of a planar dielectric sheet. A plane wave propagating along the direction $\mathbf{k}$ with electric field $\mathbf{E}$ polarized along the z direction as shown in Fig. \ref{} is incident on the dielctric surface at an angle $\phi_i$.
%
\begin{subequations}
  \begin{align}
    \v J_s &=  \hat{\v{n}} \x \v{H} = \hat {\v{z}} \mathrm{J(\zeta)},
    \label{eq:J_s}\\
    \v M_s &=  -\hat{\v{n}} \x \v{E} = \hat {\v{x}} \mathrm{M(\zeta)}
    \label{eq:M_s}
  \end{align}
  \label{eq:eq_currents}
\end{subequations}
%
where the normal unit vector $\hat{\v{n}}$ is in the y direction and $\zeta$ depends on $x$ and $y$. To find the surface currents, we set up an homogeneous equivalent problem first for the region outside the dielectric sheet as depicted in Fig. \ref{}. The total field can be written as:
%
\begin{equation}
  \v E_1 = \v E_i + \v E_1^{scat}
  \label{eq:E1}
\end{equation}
%
where $\v E_i$ is the incident electric field due to the plane wave,
%
\begin{equation}
  \v E_i = \hat{\v z} \; E^0  e^{-j k_0 (x \cos \phi_i - y \sin \phi_i)}
  \label{eq:E_i}
\end{equation}
%
with $k_0$ as the propagation constant of air and $E^0$ the amplitude of the incoming plane wave. The scattered field in (\ref{eq:E1}) can be expressed as:
%
\begin{equation}
  \v E_1^{scat} = \left( k_0^2 + \del \del \cdot \right) \v A
  - \frac{1}{\E_1} \del \x \v F
  \label{eq:E1scat}
\end{equation}
%
where $\v A$ and $\v F$ are the magnetic and electric vector potentials respectively, given by:
%
\begin{subequations}
  \begin{align}
    \v A &=  \frac{\u}{4 \pi} \iint\limits_{S} \v J_s(\v r') \frac{ e^{-j k_0 |\v r - \v r'|}}{|\v r - \v r'|} \diff{S'},
    \label{eq:A}\\
    \v F &=  \frac{\E}{4 \pi} \iint\limits_{S} \v M_s(\v r') \frac{ e^{-j k_0 |\v r - \v r'|}}{|\v r - \v r'|} \diff{S'}.
    \label{eq:Fig}
  \end{align}
  \label{eq:potentials}
\end{subequations}
%
with the position vectors $\v r$ and $\v r'$ illustrated in Fig. \ref{}. For a sheet structure extending to infiniity in the z direction, \eqref{eq:potentials} can be re-written as:
%
\begin{subequations}
  \begin{align}
    \v A &=  \frac{\u}{4 j} \int\limits_{l} \v J_s(\v \p') H_0^{(2)}(k_0 |\v \p - \v \p'|) \diff{l'},
    \label{eq:A}\\
    \v F &=  \frac{\E}{4 j} \int\limits_{l} \v M_s(\v \p') H_0^{(2)}(k_0 |\v \p - \v \p'|) \diff{l'},
    \label{eq:Fig}
  \end{align}
  \label{eq:potentials_2d}
\end{subequations}
%
where $H_0^{(2)}(k_0 |\v \p - \v \p'|)$ is the Hankel function of order 0 and the second kind. For a z-directed source, the scattered electric field in \eqref{eq:E1scat} can be simply written as:
%
\begin{equation}
  \begin{split}
    \v E_1^{scat} &= -\hat{\v z} j \O \mathrm A_z \\
    &= -\hat{\v z} \frac{\O \u}{4 j} \int\limits_{l} J_z(\v \p')  H_0^{(2)}(k_0 |\v \p - \v \p'|) \diff{l'}
  \end{split}
  \label{eq:E1sc}
\end{equation}
%
The scattered magnetic field can similarly be expressed in terms of the vector potenatials. For the case in consideration, we obtain:
%
\begin{equation}
  \begin{split}
    \v H_1^{scat} &= -\hat{\v x}  \frac{j \O}{k_0^2}\left(k_0^2 +  \pdv[2]{}{x} \right) \mathrm F_x \\
    &= -\hat{\v x}  \frac{j \O}{k_0^2}\left(k_0^2 +  \pdv[2]{}{x} \right) \int\limits_{l} M_x(\v \p') H_0^{(2)}(k_0 |\v \p - \v \p'|) \diff{l'}
  \end{split}
  \label{eq:H1sc}
\end{equation}

For the region inside the dielectric, an interior equivalent is set up with the currents reversing the signs. The total fields for the interior region only contain the scattered fields.
%
\begin{subequations}
  \begin{align}
    \v E_2^{scat} &= -\hat{\v z} \frac{\O \u}{4 j} \int\limits_{l} -J_z(\v \p')  H_0^{(2)}(k_2 |\v \p - \v \p'|) \diff{l'}
    \label{eq:E2sc}\\
    \v H_2^{scat} &= -\hat{\v x}  \frac{j \O}{k_2^2}\left(k_2^2 +  \pdv[2]{}{x} \right) \int\limits_{l} -M_x(\v \p') H_0^{(2)}(k_2 |\v \p - \v \p'|) \diff{l'}
    \label{eq:H2sc}
  \end{align}
  \label{eq:2sc}
\end{subequations}

In order to find the electric and magnetic currents, we apply the boundary conditions at the interface ensuring the continuity of tangential component of the fields. At the interface:
%
\begin{subequations}
  \begin{align}
    \hat{\v n} \x (\v E_1 - \v E_2) ={}& \v 0
    \label{eq:BC_E}\\
    \hat{\v n} \x (\v H_1 - \v H_2) ={}& \v 0
    \label{eq:BC_H}
  \end{align}
  \label{eq:BC}
\end{subequations}
%
Since the electric field is only z-directed, we obtain a scalar equation by the application of \eqref{eq:BC_E}:
%
\begin{align}
  E_i &= \frac{\O \u}{4} \int\limits_c J_z(\v \p') \left[ H_0^{(2)}(k_0 |\v \p - \v \p'|) + H_0^{(2)}(k_2 |\v \p - \v \p'|)\right] \diff{l'}
  \label{eq:scalarE}
\end{align}
%
Similarly, the magnetic field can be written as:
%
\begin{align}
  H_i &=  \frac{j \O}{k_0^2}\left(k_0^2 +  \pdv[2]{}{x} \right) \int\limits_{l} -M_x(\v \p') H_0^{(2)}(k_0 |\v \p - \v \p'|) \diff{l'} \notag\\
  &\qquad + \frac{j \O}{k_2^2}\left(k_2^2 +  \pdv[2]{}{x} \right) \int\limits_{l} -M_x(\v \p') H_0^{(2)}(k_2 |\v \p - \v \p'|) \diff{l'}
  \label{eq:scalarH}
\end{align}
%
\eqref{eq:scalarH} represents an integro-differential equation in which the differential and integral operators on the right hand side may be interchanged, thereby obtaining:
%
\begin{align}
  H_i &=  \frac{j \O}{k_0^2} \int\limits_{l} -M_x(\v \p') \left(k_0^2 +  \pdv[2]{}{x} \right) H_0^{(2)}(k_0 |\v \p - \v \p'|) \diff{l'} \notag\\
  &\qquad + \frac{j \O}{k_2^2} \int\limits_{l} -M_x(\v \p') \left(k_2^2 +  \pdv[2]{}{x} \right) H_0^{(2)}(k_2 |\v \p - \v \p'|) \diff{l'}
  \label{eq:scalarH_pock}
\end{align}
%
Operators with the order as in \eqref{eq:scalarH_pock} represent \emph{Pocklington's} integro-differential equation \cite{}. The second order derivative can be removed by expressing in terms of other Hankel functions through the recurrence relations \cite[p. 361]{}.
%
\begin{subequations}
  \begin{align}
    \dv{H_0^{(2)}(x)}{x} &= -H_{1}^{(2)}(x) + \frac{1}{x} H_0^{(2)}(x)
    \label{eq:Hankel_ID1}\\
    H_{1}^{(2)}(x)  &= \frac{x}{2} \left[H_{0}^{(2)}(x) + H_{2}^{(2)}(x)\right]
    \label{eq:Hankel_ID2}
  \end{align}
  \label{eq:Hankel_ID}
\end{subequations}
%
Furthermore, A Hankel with an argument $ k_i r = k_i|\v \p - \v \p'|$, where $i = 0,2$ can be differentiated by the chain-rule:
%
\begin{equation}
  \begin{split}
    \pdv{H_0^{(2)}(k_i r)}{x} &= \dv{H_0^{(2)}(k_i r)}{k_i r} \pdv{k_i r)}{x} \\
    &= \dv{H_0^{(2)}(k_i r)}{k_i r} \x \frac{k_i (x - x')}{r}
  \end{split}
  \label{eq:Han}
\end{equation}
%
By differentiating \eqref{eq:Han}, we obtain:
%
\begin{align}
  \pdv[2]{H_0^{(2)}(k_i r)}{x} &= \frac{k_i}{r} \left[H_2^{(2)}(k_i r) \frac{k_i (x - x')^2}{r} - H_1^{(2)}(k_i r) \right]
  \label{eq:difHan}
\end{align}
%
The differential operator in \eqref{eq:scalarH_pock} can now removed by applying the recurrence relations \eqref{eq:Hankel_ID} and the expression is rewritten as:
%
\begin{equation}
  \begin{split}
    \left(k_i^2 + \pdv[2]{}{x} \right) H_0^{(2)}(k_i r) &= \frac{k_i^2}{2} H_0^{(2)}(k_i r) + k_i^2 \left[ \frac{(x-x')^2}{r^2} - \frac{1}{2} \right] H_2^{(2)}(k_i r) \\
    &= \frac{k_i^2}{2} H_0^{(2)}(k_i r) + k_i^2 \left( \cos \zeta - \frac{1}{2} \right) H_2^{(2)}(k_i r) \\
    &= \frac{k_i^2}{2} H_0^{(2)}(k_i r) + k_i^2 \cos {(2\zeta)} H_2^{(2)}(k_i r)
  \end{split}
  \label{eq:Hankel_final}
\end{equation}
%
where $\cos \zeta = {(x-x')/r}$. The magnetic field in \eqref{eq:scalarH_pock} can be re-expressed as:
%
\begin{align}
  H_i &=  \frac{j \O}{2} \int\limits_{l} -M_x(\v \p') \left[ H_0^{(2)}(k_0 r) + \cos {(2 \zeta)} H_2^{(2)}(k_0 r) \right. \notag\\
  &\qquad \left. {} + H_0^{(2)}(k_2 r) + \cos {(2\zeta)} H_2^{(2)}(k_2 r)\right]\diff{l'}
  \label{eq:H_final}
\end{align}

\section{x-directed plate}

For
\end{document}
