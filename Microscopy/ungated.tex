Generation of Plasma waves

Talk about ungated plasma waves
dispersion relation
dyakanov-shur instability
resonance
terahertz
conductivity
mobility
dielectric function approximated


For the ungated 2DEG, the dispersion relation is given by:
\begin{equation}
  \O = \sqrt{Ne^2 /m_s}
  \label{eq:ungated_disp}
\end{equation}

The plasma frequency for an ungated region is an order of magnitude higher than the gated region of the transistor. Moreover, the Q-factor of the resonance is also higher. This paves the way for devices operating in the range of 20-30 THz. Some of the molecules found in biological sciences fluoresce in this region. Therefore, plasma oscillations that are resonant can be used to excite these molecules.

The generation of plasma waves in the gated region has been thoroughly studied owing to their analogy with the shallow water waves found in hydrodynamics. A similar treatment of plasma waves in the ungated region are analagous to deep water waves {\cite Shur ungated}.

When the plasma waves come across appropriate boundary conditions in the form of an ac open short source and and dc short drain terminal, they tend to reflect and form a standing wave in the 2DEG. Further excitement of these plasma waves lead to their instability. For an ungated region, due to the absence of grating structure in the form of the gate terminal, the plasma waves tend to not radiate as efficiently due to the momentum mismatch (large difference of wavenumber) between the plasma waves and free-space. However, due to the very thin nature of the superstatet spacer layer on the 2DEG, partially decayed plasma waves exist at the top face of the structure which can be used to excite the sample.

In order to obtain a strong standing wave pattern, the device should be operated near the plasma frequency so that the dielectric function of the 2DEG region is vanishingly small. Furthermore, the real part of the conductivity is much smaller than its imaginary part. The conductivity is given by:

\begin{equation}
    \sigma_s(\O) = \frac{N_s e^2}{m_{\ast}} \tau \frac{1}{1 + j \O \tau}
    \label{eq:conductivity}
\end{equation}

The quantity $\tau$ is the mean scattering time of the electrons in the 2DEG which is expressed in terms of the electron mobility:

\begin{equation}
  \tau = \frac{\u m_{\ast}}{e}
  \label{eq:tau}
\end{equation}

In order to obtain the desired properties for conductivity and dielectric function discussed above, the mobility should be as high as possible. Unfortunately, it is highly temperature dependent and reduces exponentially with increase in temperature by the following relation:

\begin{equation}
  \u \prop \frac{1}{T^{3/2}}
  \label{eq:uT}
\end{equation}

Thus the device must be cooled to cryogenic temperatures. Moreoever if the product $\O \tau >> 1$, the oscillation in the 2DEG are undamped.

For thin structures, the sheet conductivity can be converted into an approximate volumetric form by the multplying with the thickness of the 2DEG layer. The resulting dielectric function is then written as:

\begin{equation}
  \E(\O) = \E_0 \E_r + j\frac{\sigma_s(\O)}{\O t \E_0}
\end{equation}

Write about the dielectric function
Dispersion relation
Simulation

GaN/AlGaN heterostructure
