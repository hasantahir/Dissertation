\documentclass[11pt]{article}

% Insert style guide
\usepackage{my_thesis}
% Specifiy the location of images to be used
\graphicspath{{figures/}}

\begin{document}
\title{\textsc{Plasma based Structured Illumination Microscopy}\\}
\date{\footnote{Last Modified: \currenttime, \today.}}
\maketitle

\section{Theory}

A two-dimensional electron gas (2DEG) is an extremely thin region with a high concentration of electrons that are free to move along the interface of the semiconductor hetero-junctions found in field-effect transistors, leading to ballistic transport of electrons owing to high mobility.

Yahan ye batana ha ke 2deg surface waves support kerti ha. kaisay batana ha yeh nahi pata.

Aik tareeqa to yeh ha ke ap ye keh do ke since the 2DEG can be considered a sheet of charge sandwiched between dieelctric media, it supports surface waves. the conductivity is given as:
%
\begin{equation}
  \sigma_s(\O) = \frac{N_s e^2 \tau_{p}}{m^{\ast}}\frac{1}{1 + j \O \tau_{p}}
  \label{eq:conductivity}
\end{equation}
%
The
\begin{equation}

\end{equation}

\section{Imaging Technique}


%
where $\tau_p$ is the momentum relaxation time and $m^{\ast}$ is the effective  mass of electrons. When the

\clearpage % Force Bibliography to the end of document on a new page
\bibliography{zubairy}
\bibliographystyle{ieeetr}

\end{document}
